\documentclass{patmorin}
\listfiles
\usepackage[utf8]{inputenc}
\usepackage{microtype}
\usepackage{amsthm,amsmath,graphicx}
\usepackage{pat}
\usepackage[letterpaper]{hyperref}
\usepackage[table,dvipsnames]{xcolor}
\definecolor{linkblue}{named}{Blue}
\hypersetup{colorlinks=true, linkcolor=linkblue,  anchorcolor=linkblue,
citecolor=linkblue, filecolor=linkblue, menucolor=linkblue,
urlcolor=linkblue} 
\setlength{\parskip}{1ex}
\usepackage{wasysym}

\title{\MakeUppercase{Notes on Local Pathwidth (for Planar Graphs)}}

\author{\ } %The Layered $\star$-width Gang}%

%\usepackage[mathlines]{lineno}
%\linenumbers
%\setlength{\linenumbersep}{2.5cm}
%\rightlinenumbers
%\linenumbers
%\newcommand*\patchAmsMathEnvironmentForLineno[1]{%
%  \expandafter\let\csname old#1\expandafter\endcsname\csname #1\endcsname
%  \expandafter\let\csname oldend#1\expandafter\endcsname\csname end#1\endcsname
%  \renewenvironment{#1}%
%     {\linenomath\csname old#1\endcsname}%
%     {\csname oldend#1\endcsname\endlinenomath}}% 
%\newcommand*\patchBothAmsMathEnvironmentsForLineno[1]{%
%  \patchAmsMathEnvironmentForLineno{#1}%
%  \patchAmsMathEnvironmentForLineno{#1*}}%
%\AtBeginDocument{%
%\patchBothAmsMathEnvironmentsForLineno{equation}%
%\patchBothAmsMathEnvironmentsForLineno{align}%
%\patchBothAmsMathEnvironmentsForLineno{flalign}%
%\patchBothAmsMathEnvironmentsForLineno{alignat}%
%\patchBothAmsMathEnvironmentsForLineno{gather}%
%\patchBothAmsMathEnvironmentsForLineno{multline}%
%}


\newcommand{\question}[1]{\textbf{\color{red}Question:}~#1}

\DeclareMathOperator{\degree}{deg}


\DeclareMathOperator{\tr}{tr}
\DeclareMathOperator{\pw}{pw}
\DeclareMathOperator{\lpw}{lpw}
\DeclareMathOperator{\wlpw}{wlpw}

\DeclareSymbolFont{sfoperators}{OT1}{cmss}{m}{n}
\DeclareSymbolFontAlphabet{\mathsf}{sfoperators}

%\makeatletter
%\def\operator@font{\mathgroup\symsfoperators}
%\makeatother

%\pagenumbering{roman}
\begin{document}
%\begin{titlepage}
\maketitle

\section{The Proof}
\pagenumbering{arabic}

This draft proves that $(2)\rightarrow (4)$ for families of planar graphs.  
There might be a shorter proof of this, but I'm hoping this one has some 
hope of generalizing.


\begin{lem}\lemlabel{dual-diameter}
   Let $G$ be an embedded planar graph of treewidth at most $k$ and let
   $G'$ be the dual graph of $G$.  Then the diameter of $G'$ is $O(k)$.
\end{lem}

\begin{proof}[Proof Sketch]
   Since $G$ has no $k\times k$ grid minor, every face is within distance
   $O(k)$ of the outer face.
\end{proof}


\begin{lem}\lemlabel{planar}
  Let $G$ be an embedded planar graph whose dual graph has diameter at most
  $k$. Then there exists constants $\beta=\beta(k)$ and $\gamma=\gamma(k)$
  such that $G$ has a width-$\beta$ tree decomposition $(T,B)$ with
  the property that, for each $v\in V(G)$, the pathwdith of $T[v]$
  is at most $\gamma$.
\end{lem}

\begin{proof}
  Without loss of generality, we may assume that $G$ is connected.
  Let $T_{G'}$ be a breadth-first search tree of $G'$ rooted at the
  outer face and let $X$ denote the set of edges in $G$ that are dual
  to edges in $T_{G'}$. Note that $T_G:=G-X$ is a spanning tree of $G$.

  Now, construct a graph $H$ with maximum degree 3 by replacing each
  vertex of degree $d\ge 3$ in $G$ with a path of length $d-3$ and let
  $f:V(H)\to V(G)$ be the (many-to-one) mapping from each vertex $v'$
  in $H$ to the vertex $v$ in $G$ that generated $v'$. Now, each
  edge $vw$ in $G$ has exactly one corresponding edge $g(vw):=v'w'$
  in $H$ where $v'\in f^{-1}(v)$ and $w'\in f^{-1}(w)$.  Let $T$ be the
  spanning tree of $H$ defined as $T=H-\{g(vw):vw\in X\}$.

  We will use $T$ (the spanning tree of $H$) in a tree decomposition
  $(T,B)$ of $G$.  Now we describe the contents of the bags $B_{v'}$ for
  each $v'\in V(T)$.  We do this in three steps:

  \begin{enumerate}
    \item For each vertex $v'$ in $T$, the tree decomposition
  includes $v=f(v')$ in the bag $B_{v'}$.  Note that, as of this point, 
  $T[v]$ is a path, $P_v$ of length $\max\{1,\deg(v)-3\}$. 

    \item Each edge $vw$ in $T_G$ has has a corresponding edge $v'w'=g(vw)$
  in $T$, so we add both $v$ and $w$ to the bags $B_{v'}$ and $B_{w'}$.
  At this point, $(T,B)$ is a tree decomposition of $T_G$ and, for each
  vertex $v$ of $G$,  $T[v]$ is a caterpillar.  

    \item For each edge $vw$
  of $X$, consider the pair of vertices $v'$ and $w'$ with $v'w'=g(vw)$.
  To complete our tree decomposition of $G$, we add one of the two
  vertices, say $v$, to the bag $B_{x'}$ for every vertex $x'$ on the
  path from from $v'$ to $w'$ in $T$.
  \end{enumerate}
  At this point, $(T,B)$ is a tree decomposition of $G$. What remains
  is to show that $(T,B)$ has small width and that, for each vertex $v$
  of $G$, $T[v]$ has small pathwidth.

  First, let us consider the width of $(T,B)$.  For any bag $v'$, at most
  4 vertices are added in Steps 1 and 2 (recall that $T$ has maximum
  degree 3).  To account for edges added in Step~3 note that, for each
  face $f$ of $H$ incident to $v'$, the bag $B_{v'}$ contains one vertex
  for each ancestor of $f$ in the dual BFS tree $T'$.  There are at most
  three faces of $H$ incident to $v'$, and, since $G'$ has diameter at
  most $k$, this contributes at most $3k$ additional vertices to $B_{v'}$.
  Thus, the width of the tree decomposition $(T,B)$ is at most $3k+4$.

  Next, we show that, for each vertex $v$ of $G$, $T[v]$ has small pathwidth.  
  The subtree $T[v]$ contains a path $P_v$ of length
  $\max\{1,\degree(v)-3\}$ whose vertices are those in $f^{-1}(v)$.
  Now, consider some component $C$ of $T[v]-P_v$.  The component $C$ is
  incident, in $T$, to exactly one vertex $v'$ of $P_v$.  Consider the
  graph $T_{C}=T[v][\{v'\}\cup V(C)]$, which is a subtree of $T[v]$ rooted
  at $v'$ and each of whose leaves is incident, in $H$, to a vertex of
  $P_v$.  We claim that $T_C$ has $\ell \le 4k$ leaves. If this were not
  the case, then we could choose a leaf $w'$ of $T_C$ that is adjacent,
  in $H$ to a vertex $v''$ in $P_v$ such that the addition of $w' v''$
  to $T[v]$ creates a cycle $S$ that contains at least $\ell/2-1$ leaves
  of $T[v]$ in its interior.  This cycle $S$ contains at least $\ell/2$
  faces of $H$ in its interior.  Each of these faces has a corresponding
  face in $G$ and in the dual BFS tree, $T_{G'}$, these faces belong to
  a common path.  But this is a contradiction if $\ell/2 > 2k$, since
  the longest path in $T_{G'}$ has length at most $2k$.

  Thus, $T[v]$ consists of path $P_v$ and each vertex of $P_v$ is the
  root of up to three trees, each having at most $4k$ leaves.  Each of
  these subtrees has a path decomposition of width at most $\log_2(4k)+1$
  in which the root is in every bag. Concatenating these decompositions
  yields a path decomposition of $T[v]$ of width at most $\log_2(4k+1)$. 
\end{proof}

\begin{thm}\thmlabel{main}
  Let $\mathcal{G}$ be a minor-closed family of planar graphs in which
  no member contains a $Q_{k}$ minor.  There there exists constants
  $\beta=\beta(k)$ and $\gamma=\gamma(k)$ such that, every $G\in
  \mathcal{G}$ has a width-$\beta$ tree decomposition $(T,B)$ with the
  property that, for each $v\in V(G)$, the pathwdith of $T[v]$ is at
  most $\gamma$.
\end{thm}


\begin{proof}
  The $k2^k\times k2^k$ grid contains a $Q_k$ minor.
  Since $G$ does not contain a $Q_{k}$ minor, it therefore does not contain
  a $k2^k\times k2^k$ grid minor and therefore it as treewidth upper bounded
  as a function of $k$.
  Therefore, by \lemref{dual-diameter}, the dual $G'$ of $G$ has constant
  diameter.  Now apply \lemref{planar}.
\end{proof}


\section{Remarks}

Of course, we would like to prove \thmref{main} without the planarity
assumption.  For this, we need something to replace the notion of the
dual graph $G'$.  Here, the purpose of the dual graph is only to find a
set of edges $X$ whose removal turns $G$ into a tree.  The low diameter
of $G'$ is used to ensure that the tree decomposition has small width
and that the subtrees $T[v]$ have low pathwidth.  Instead, we should be
looking for an $X$ directly that has the properties we want.

\bibliographystyle{plain}
\bibliography{localpathwidth}

\end{document}


