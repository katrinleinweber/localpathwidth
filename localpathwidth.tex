\documentclass{patmorin}
\listfiles
\usepackage[utf8]{inputenc}
\usepackage{microtype}
\usepackage{amsthm,amsmath,graphicx}
\usepackage{pat}
\usepackage[letterpaper]{hyperref}
\usepackage[table,dvipsnames]{xcolor}
\definecolor{linkblue}{named}{Blue}
\hypersetup{colorlinks=true, linkcolor=linkblue,  anchorcolor=linkblue,
citecolor=linkblue, filecolor=linkblue, menucolor=linkblue,
urlcolor=linkblue} 
\setlength{\parskip}{1ex}
\usepackage{wasysym}

\title{\MakeUppercase{Notes on Local Pathwidth}}

\author{\ } %The Layered $\star$-width Gang}%

%\usepackage[mathlines]{lineno}
%\linenumbers
%\setlength{\linenumbersep}{2.5cm}
%\rightlinenumbers
%\linenumbers
%\newcommand*\patchAmsMathEnvironmentForLineno[1]{%
%  \expandafter\let\csname old#1\expandafter\endcsname\csname #1\endcsname
%  \expandafter\let\csname oldend#1\expandafter\endcsname\csname end#1\endcsname
%  \renewenvironment{#1}%
%     {\linenomath\csname old#1\endcsname}%
%     {\csname oldend#1\endcsname\endlinenomath}}% 
%\newcommand*\patchBothAmsMathEnvironmentsForLineno[1]{%
%  \patchAmsMathEnvironmentForLineno{#1}%
%  \patchAmsMathEnvironmentForLineno{#1*}}%
%\AtBeginDocument{%
%\patchBothAmsMathEnvironmentsForLineno{equation}%
%\patchBothAmsMathEnvironmentsForLineno{align}%
%\patchBothAmsMathEnvironmentsForLineno{flalign}%
%\patchBothAmsMathEnvironmentsForLineno{alignat}%
%\patchBothAmsMathEnvironmentsForLineno{gather}%
%\patchBothAmsMathEnvironmentsForLineno{multline}%
%}


\newcommand{\question}[1]{\textbf{\color{red}Question:}~#1}

\DeclareMathOperator{\degree}{deg}


\DeclareMathOperator{\tr}{tr}
\DeclareMathOperator{\pw}{pw}
\DeclareMathOperator{\lpw}{lpw}
\DeclareMathOperator{\wlpw}{wlpw}

\DeclareSymbolFont{sfoperators}{OT1}{cmss}{m}{n}
\DeclareSymbolFontAlphabet{\mathsf}{sfoperators}

%\makeatletter
%\def\operator@font{\mathgroup\symsfoperators}
%\makeatother

%\pagenumbering{roman}
\begin{document}
%\begin{titlepage}
\maketitle

\section{Introduction}
\pagenumbering{arabic}

\begin{lem}\lemlabel{dual-diameter}
  Let $G$ be an embedded planar graph of treewidth at most $k$ and let
  $G'$ be the dual graph of $G$.  Then the diameter of $G'$ is $O(k)$.
\end{lem}


\begin{lem}\lemlabel{planar}
  Let $G$ be an embedded planar graph whose dual graph has diameter at most
  $k$. Then there exists constants $\beta=\beta(k)$ and $\gamma=\gamma(k)$
  such that $G$ has a width-$\beta$ tree decomposition $(T,B)$ with
  the property that, for each $v\in V(G)$, the pathwdith of $T[v]$
  is at most $\gamma$.
\end{lem}

\begin{proof}
  Without loss of generality, we may assume that $G$ is connected.
  Let $T_{G'}$ be a breadth-first search tree of $G'$ rooted at the
  outer face and let $X$ denote the set of edges in $G$ that are dual
  to edges in $T_{G'}$. Note thet $T_G:=G-X$ is a spanning tree of $G$.

  Now, construct a graph $H$ with maximum degree 3 by replacing each
  vertex of degree $d\ge 3$ in $G$ with a path of length $d-3$ and let
  $f:V(H)\to V(G)$ be the (many-to-one) mapping from each vertex $v'$
  in $H$ to the vertex $v$ in $G$ that generated $v'$. Similarly, each
  edge $vw$ in $G$ has exactly one corresponding edge $g(vw):=v'w'$
  in $H$ where $v'\in f^{-1}(v)$ and $w'\in f^{-1}(w)$.  Let $T$ be the
  spanning tree of $H$ defined as $T=H-\{g(vw):vw\in X\}$.

  We will use $T$ (the spanning tree of $H$) in a tree decomposition
  $(T,B)$ of $G$ For each $v'$ in $T$, our tree decomposition includes
  $v=f(v')$ in the bag $B_{v'}$.  For each edge $vw$ of $T_G$,

  Each edge $vw$ in $T_G$ has has a corresponding edge $v'w'=g(vw)$
  in $T$, so we add both $v$ and $w$ to the bags $B_{v'}$ and $B_{w'}$.
  By now, $(T,B)$ is a tree decomposition of $T_G$.  What remains is to
  include  the edges of $X$.  For each edge $vw$ of $X$, consider
  the pair of vertex $v'$ and $w'$ with $v'w'=g(vw)$.  To complete our
  tree decomposition of $G$, we add one of the two vertices, say $v$,
  to the bag $B_{x'}$ for every vertex $x'$ on the path from from $v'$
  to $w'$ in $T$.

  At this point, $(T,B)$ is a tree decomposition of $G$. What remains
  is to show that $(T,B)$ has small width and that, for each vertex $v$
  of $G$, $T[v]$ has small pathwidth.

  First, consider $T[v]$ for some vertex $v$ of $G$. Initially, $T[v]$
  starts out as a path $P_v$ of length $\max\{1,\degree(v)-3\}$.  Later,
  for each edge $vw$ in $X$, $T[v]$ expands to include a path from $v'$
  to $w'$.  Nevertheless, we claim that $T[v]$ is a tree all of whose
  branching nodes lie on the common path $P_v$, i.e., $T[v]$ is a
  subdivision of a caterpillar.
\end{proof}

\begin{thm}
  Let $\mathcal{G}$ be a minor-closed family of planar graphs in which
  each member has layered pathwidth at most $\alpha$.
  There there exists constants $\beta=\beta(\alpha)$ and $\gamma=\gamma(\alpha)$
  such that, every $G\in \mathcal{G}$ has a width-$\beta$ tree decomposition
  $(T,B)$ with the property that, for each $v\in V(G)$, the pathwdith
  of $T[v]$ is at most $\gamma$.
\end{thm}


\begin{proof}
  Recall that $T_{4\alpha+1}$ has pathwidth greater than $2\alpha$,
  which implies that $Q_{4\alpha+1}$ has layered pathwidth greater than
  $\alpha$.  This implies that $G$ does not contain a $Q_{4\alpha+1}$
  minor, so the treewidth of $G$ is $O(1)$.  Therefore, by
  \lemref{dual-diameter}, the dual $G'$ of $G$ has constant diameter.
  Now apply \lemref{planar}.
\end{proof}

\bibliographystyle{plain}
\bibliography{localpathwidth}

\end{document}


